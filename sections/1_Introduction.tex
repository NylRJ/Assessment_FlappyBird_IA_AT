\section{Introdução}\label{sec:intro}
A Natureza sempre serviu de inspiração para diversas áreas, no campo artístico temos pintores, poetas e na arquitetura podemos citar a resolução do problema do \textbf{Trem-bala} do \textbf{Japão}. A solução do problema foi encontrada por \textbf{Eiji Nakatsu}, engenheiro e observador de pássaros, tentando contornar o problema da poluição sonora e a explosão sônica, ele teve a ideia de  imitar a aerodinâmica do pássaro \textbf{Martim-pescador (um espécie pássaro)}.  A ave, que precisa mergulhar para se alimentar, troca rapidamente de um ambiente de baixa resistência (ar) para um com muita resistência (água), ela possui a aerodinâmica perfeita para essa situação. Por isso o Trem-bala foi inspirado nesta ave\cite{area_engenaria} Na tecnologia da informação (TI) não é diferente, tivemos \textbf{Alan Kay}, ele era matemático, biólogo, criou o paradigma de orientação a objetos.  O matemático, formulou sua "analogia algébrico-biológica". \textbf{Kay}, lançou o postulado de que segundo ele, “o computador ideal deveria funcionar como um organismo vivo, isto é, cada célula se relaciona com outras a fim de alcançar um objetivo, mas cada uma funciona de forma autônoma. As células poderiam também reagrupar-se para resolver outro problema, ou desempenhar outras funções”.\cite{Alan_Kay}
Em virtude disso, o conhecimento das unidades principais do cérebro, conhecidas como os neurônios, estabelecem explicações para iniciar a compreensão deste estudo.
\cite{Neuronio}.
  O funcionamento do cérebro consiste em desenvolve suas regras através da experiência adquirida em situações vividas anteriormente\cite{Neuronio}.
A busca pelo desenvolvimento de um programas que realize funções da mente humana
começou logo  partir de 1930, começam as pesquisas para substituir as partes mecânicas por elétricas, como exemplo o \textbf{Mark I} (A Calculadora Automática de Seqüência Controlada), concluído em 1944\cite{uff}. Desde então, muitas pesquisas foram
feitas para chegar a este objetivo. Na tentativa, surgem modelos computacionais interessantes, dentro da grande área de Inteligência Artificial.
As Redes Neurais Artificiais, também designadas neste trabalho por RNA, que são inspiradas no funcionamento do cérebro humano que usa uma abstração matemática do neurônio biológico denominada de perceptron.\cite{MCCULLOCH}
Com a redução do custo de equipamentos utilizados
para processar redes neurais, oportunizam a execução de algoritmos de aprendizado em
computadores pessoais e dando novo fomento ao desenvolvimento de novos algoritmos, e softwares que utilizam \textbf{RNA}.
As Redes neuronais artificiais são modelos computacionais inspirados pelo sistema nervoso central, eles são capazes de realizar o aprendizado de máquina, bem como o reconhecimento de padrões,  para criar esses modelos ou métodos avançados  requer um grande conhecimento matemático e demanda de muito tempo. Tendo em vista que hoje focamos no desenvolvimento de software ágil, seria contraproducente perder muito tempo no desenvolvimento desses modelos do zero.
Porém, com o surgimento das bibliotecas, TensorFlow, Keras e Neat-pytho,	
esse processo se torna muito mais fácil e nos ajuda a construir e implementar facilmente  modelos de aprendizagem de máquina. Neste projeto usarei o
Neat-python.
O NEAT, que tem como objetivo criar uma rede neural artificial que evolui através de mudança em sua delineação e alteração de pesos das suas conexões, seguindo a abordagem de algoritmos neuroevolutivos.
O \textbf{NEAT} é um algoritmos de neuroevolução que melhora o desempenho da arquiteturas de rede neural.
Este trabalho consiste em utilizar a \textbf{programação Orientada a objeto (POO)} a biblioteca \textbf{NEAT-PYTHON} para evolução de redes neurais arbitrárias, tendo como foco
criar o jogo \textbf{Flappy Bird}, por ser um jogo simples e de fácil implementação, facilitando a compreenção de algoritmos de neuroevolução para melhorar o desempenho de arquiteturas de rede neural.
Os objetivos específicos deste trabalho são: \break 
\begin{itemize}
    \item Estudar conceitos básicos
de evolução de redes neurais;\\
     \item Mapear soluções de software pré-existentes para a montagem do sistema de aprendizado de neuroevolução;\\
     \item Utilizarar a biblioteca do NEAT-python para a criação do jogo Flappy Bird.\\
     \item Avaliar os resultados obtidos e discutir sobre seu desempenho.
\end{itemize} 


