\section{Observações Finais}\label{sec:conclusion}
Neste Trabalho foi utilizado Neuroevolução com a biblioteca \textbf{Neat-Python}, para implementação do jogo Flappy Bird, trabalho inspirado em vários outros trabalhos.\cite{Cheesy}\cite{CodeBucket}\cite{CodeBucket}\cite{GamePlaying}
Os quais me ajudaram a entender como utilizar as bibliotecas do\textbf{Neat-Python, Pygame}, e a compreensão de parâmetros de configuração da rede do\textbf{Neat-Python}
No primeiro teste a realizado com uma arquitetura de redes neurais,
os resultados pareciam muito bons.
Na segunda geração o agente já havia alcançado seu objetivo
Na segunda parte do experimento houve uma demora na convergência,
percebia-se bons parâmetros na primeira geração, no entanto depois de 36 gerações
o agente não alcançou o seu objetivo e o algoritmo foi finalizado na trigésima sétima geração.
Após rodar o código novamente o agente na primeira geração já havia alcançado o seu objetivo, porém o ponto mais incrível foram seus movimentos que ficaram muito suave
parecia uma queda de paraquedas.
Então pude consta que que no primeiro teste  apesar de uma rede simples ele ele obteve bons resultados seus movimentos me pareciam bons até compará lo com o segundo
após rodar lo pela 2 vez
A minha conclusão é que o segundo teste te entregar resultados ótimos porém com um custo de tempo e processamento e o primeiro teste te entregar resultados bons com baixo custo de tempo e processamento.


Meus agradecimentos ao meu professor por me exaurir até o máximo e arrancar de mim bons resultados na produção deste relatório que só me agregou conhecimento.


